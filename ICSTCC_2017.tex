% $Header$

\documentclass{beamer}

% This file is a solution template for:

% - Talk at a conference/colloquium.
% - Talk length is about 20min.
% - Style is ornate.



% Copyright 2004 by Till Tantau <tantau@users.sourceforge.net>.
%
% In principle, this file can be redistributed and/or modified under
% the terms of the GNU Public License, version 2.
%
% However, this file is supposed to be a template to be modified
% for your own needs. For this reason, if you use this file as a
% template and not specifically distribute it as part of a another
% package/program, I grant the extra permission to freely copy and
% modify this file as you see fit and even to delete this copyright
% notice. 


%\mode<presentation>
%{
%  \usetheme{Warsaw}
%  % or ...
%
%  \setbeamercovered{transparent}
%  % or whatever (possibly just delete it)
%}
%
%
%\usepackage[english]{babel}
%% or whatever
%
%\usepackage[latin1]{inputenc}
%% or whatever
%
%\usepackage{times}
%\usepackage[T1]{fontenc}
%% Or whatever. Note that the encoding and the font should match. If T1
%% does not look nice, try deleting the line with the fontenc.




% Setup appearance:
\mode<presentation>
{
\usetheme{Darmstadt}
\usefonttheme[onlylarge]{structurebold}
\setbeamerfont*{frametitle}{size=\normalsize,series=\bfseries}
\setbeamertemplate{navigation symbols}{}
\usefonttheme{professionalfonts}
}

% Standard packages

\usepackage[english]{babel}
\usepackage[latin1]{inputenc}
\usepackage{times}
\usepackage[T1]{fontenc}

%
%\usefonttheme[onlymath]{serif}


% Setup TikZ

\usepackage{tikz}
\usetikzlibrary{arrows}
\tikzstyle{block}=[draw opacity=0.7,line width=1.4cm]




\title[AUV control without linear velocity measurements] % (optional, use only with long paper titles)
{Inertial-aided Homography-based Visual Servo Control of Autonomous Underwater Vehicles without Linear Velocity Measurements}

%\subtitle
%{Include Only If Paper Has a Subtitle}

%\author[Author, Another] % (optional, use only with lots of authors)
%{F.~Author\inst{1} \and S.~Another\inst{2}}
% - Give the names in the same order as the appear in the paper.
% - Use the \inst{?} command only if the authors have different
%   affiliation.

%\institute[Universities of Somewhere and Elsewhere] % (optional, but mostly needed)
%{
%  \inst{1}%
% Department of Computer Science\\
% University of Somewhere
% \and
%  \inst{2}%
%  Department of Theoretical Philosophy\\
%  University of Elsewhere}
% - Use the \inst command only if there are several affiliations.
% - Keep it simple, no one is interested in your street address.

\author[Nguyen, Hua, Allibert and Hamel] % (optional, use only with lots of authors)
{L.-H.~Nguyen \and M.-D.~Hua \and G.~Allibert \and T.~Hamel}
\institute[Universities of Somewhere and Elsewhere] % (optional, but mostly needed)
{
	\textit{Universit\'e C\^ote d'Azur, CNRS, I3S}\\
	Sophia Antipolis, France \\
	lhnguyen(hua,allibert,thamel)@i3s.unice.fr}
	
	
\date[ICSTCC 2017] % (optional, should be abbreviation of conference name)
{21st ICSTCC, 2017}
% - Either use conference name or its abbreviation.
% - Not really informative to the audience, more for people (including
%   yourself) who are reading the slides online

%\subject{Theoretical Computer Science}
% This is only inserted into the PDF information catalog. Can be left
% out. 



% If you have a file called "university-logo-filename.xxx", where xxx
% is a graphic format that can be processed by latex or pdflatex,
% resp., then you can add a logo as follows:

% \pgfdeclareimage[height=0.5cm]{university-logo}{university-logo-filename}
% \logo{\pgfuseimage{university-logo}}



% Delete this, if you do not want the table of contents to pop up at
% the beginning of each subsection:
\AtBeginSubsection[]
{
  \begin{frame}<beamer>{Outline}
    \tableofcontents[currentsection,currentsubsection]
  \end{frame}
}


% If you wish to uncover everything in a step-wise fashion, uncomment
% the following command: 

%\beamerdefaultoverlayspecification{<+->}


\begin{document}

\begin{frame}
  \titlepage
\end{frame}

\begin{frame}{Outline}
  \tableofcontents
  % You might wish to add the option [pausesections]
\end{frame}


% Structuring a talk is a difficult task and the following structure
% may not be suitable. Here are some rules that apply for this
% solution: 

% - Exactly two or three sections (other than the summary).
% - At *most* three subsections per section.
% - Talk about 30s to 2min per frame. So there should be between about
%   15 and 30 frames, all told.

% - A conference audience is likely to know very little of what you
%   are going to talk about. So *simplify*!
% - In a 20min talk, getting the main ideas across is hard
%   enough. Leave out details, even if it means being less precise than
%   you think necessary.
% - If you omit details that are vital to the proof/implementation,
%   just say so once. Everybody will be happy with that.

\section{Motivation}

%\subsection{The Basic Problem}

\begin{frame}{Why Visual Servo Control?}
  % - A title should summarize the slide in an understandable fashion
  %   for anyone how does not follow everything on the slide itself.
%	\begin{figure}[ht!]
	%\centering
%		\includegraphics[width=30mm]{Images/ROV_operations_PRODUCT_IMAGE2.jpg}
%		\includegraphics[width=30mm]{Images/ROV_operations_PRODUCT_IMAGE1.jpg}
%		\caption{ROV and its operation, source: www.jfdglobal.com}
%	\label{ROV}
%\end{figure}




	\begin{columns}[t]
		\column{.5\textwidth}	
		\begin{block}{Global acoustic positioning system issues:}		
			\begin{itemize}
				\item Unusable
				\item Insufficiently precise 
			\end{itemize}
		\end{block}
		\begin{block}{Camera}		
			\begin{itemize}
				\item Allow operation closed to structures
				\item Rich of informations
			\end{itemize}
		\end{block}
	
		
		\column{.5\textwidth}		
		\begin{block}{Stereo camera}
			\begin{itemize}
				\item High processing power required
				\item Ineffective when far from scene
			\end{itemize}
		\end{block}
		\begin{block}{Monocular camera}
			\begin{itemize}
				\item Impossible pose reconstruction   
			\end{itemize}
		\end{block}
	\end{columns}

\end{frame}

\begin{frame}{Linear velocity measurement issues}
	\begin{columns}[t]
		\column{.3\textwidth}
		\begin{block}{DVL issues:}
			\begin{itemize}
				\item High price ($>$ 10K US\$)
				\item High weight
				%\item Big dimensions 
				\item Violate maximum slope-threshold of DVLs in close proximity operation
			\end{itemize}
		\end{block}
		\column{.8\textwidth}
		\begin{figure}
			\includegraphics[width = 80mm]{Images/DVL_prices.png}\\ \vspace{0.2cm}
			 DVL comparison, \tiny from Terzja van de Kuil, 2016
		\end{figure}
		
	\end{columns} 
\end{frame}

\section{System modeling \& Control design}

\begin{frame}{System modeling}
	\begin{columns}
		\column[content...]{.5\textwidth}
		\begin{figure}
			\includegraphics[width = 55mm]{Images/Notation.png}
		\end{figure}
		\column[placement]{.6\textwidth}
		\begin{block}{Equations of motion:}
			\scriptsize
			\begin{equation*}\label{eq:system2}
			\begin{array}{rl}
			\dot{\mathbf{p}} & =  \mathbf{R} \mathbf{V} \\
			\dot{\mathbf{R}} & =  \mathbf{R} \mathbf{\Omega}_\times \\
			\!\!\dot{\mathbf{P}}_h &= \mathbf{P}_h \times \mathbf{\Omega} +\mathbf{F}_c + \mathbf{F}_{gb} + \mathbf{F}_d  \\
			\!\!\dot{\mathbf{\Pi}}_h &=  \mathbf{\Pi}_h \!\times\! \mathbf{\mathbf{\Omega}} + \mathbf{P}_h \!\times \!\mathbf{V}_h  +\mathbf{\Gamma}_c + \mathbf{\Gamma}_g + \mathbf{\Gamma}_d \!\!\vspace{-0.2cm}
			\end{array}
			\end{equation*} \vspace{-0.3cm}
			where: \\
			\begin{equation*}
			\begin{array}{rl}
			\mathbf{P}_h & = \mathbf{M}\mathbf{V}_h + \mathbf{D}^{\!\top} \mathbf{\Omega}\\[1ex]
			\mathbf{\Pi}_h & = \mathbf{J} \mathbf{\Omega} + \mathbf{D}\mathbf{V}_h \\
			\mathbf{F}_{gb}  &\triangleq (m g - F_b) \mathbf{R}^{\!\top} \mathbf{e}_3\\
			\mathbf{\Gamma}_g  &\triangleq mgl\mathbf{e}_{3}\! \times \!\mathbf{R}^{\!\top} \mathbf{e}_3\\
			\mathbf{F}_d(\mathbf{V}_h) & \triangleq - (\mathbf{D}_{V\!l}   +|\mathbf{V}_h|\mathbf{D}_{V\!q} )\mathbf{V}_h  \\
			\mathbf{\Gamma}_d(\mathbf{\Omega})& \triangleq  - (\mathbf{D}_{\Omega l}  + |\mathbf{\Omega}|\mathbf{D}_{\Omega q} )\mathbf{\Omega}
			\end{array}
			\end{equation*}
		\end{block}
	\end{columns}

\end{frame}


\begin{frame}{Model for control design}
\begin{columns}
	\column[content...]{0.4\textwidth}
	\begin{block}{Simplifications:}
		\begin{itemize}
			\scriptsize
			\item Coupling matrix $\mathbf{D}$
			\begin{itemize}
				\scriptsize
				\item Compact-shape AUVs
				\item Small distance BG
				\item Light weight AUVs
			\end{itemize}
			\item "Munk moment" $\mathbf{P}_h \!\times \!\mathbf{V}_h$
		\end{itemize}		
	\end{block}
	\begin{block}{Compact-shape AUV}
		\begin{figure}
			\includegraphics[width = 1.0\textwidth]{Images/Bluerov.png}\\
			\tiny Source: www.bluerobotics.com
		\end{figure}
	\end{block}
	\column[content...]{0.68\textwidth}
	\begin{block}{Model for control design}
		\scriptsize
		\[
		\begin{array}{rl}
		\dot{\mathbf{p}} & =  \mathbf{R} \mathbf{V} \label{eq:kinematicsPos}\\
		\dot{\mathbf{R}} & =  \mathbf{R} \mathbf{\Omega}_\times \\
		\mathbf{M}\dot{\mathbf{V}} &= (\mathbf{M}\mathbf{V}) \!\times\!\mathbf{\Omega} + \mathbf{F}_c + \mathbf{F}_{gb} + \mathbf{F}_d(\mathbf{V}) +\mathbf{\Delta}_F  \\
		\mathbf{J}\dot{\mathbf{\Omega}} &=  (\mathbf{J}\mathbf{\Omega}) \!\times\! \mathbf{\mathbf{\Omega}} + \mathbf{\Gamma}_c + \mathbf{\Gamma}_g +  \mathbf{\Delta}_{\Gamma}
		\end{array} 
		\]
		
		
		\noindent with the ``disturbance'' terms: 
		\[
		\begin{array}{rl}
		\mathbf{\Delta}_F \triangleq\!\!\!\!\!\!& -(\mathbf{M}\mathbf{V}_f)_\times \mathbf{\Omega} -\mathbf{M} \mathbf{\Omega}_\times \mathbf{V}_f + (\mathbf{D}^\top \mathbf{\Omega})_\times \mathbf{\Omega} -\mathbf{D}^\top \dot{\mathbf{\Omega}}\\
		&+ \mathbf{F}_d(\mathbf{V}_h) - \mathbf{F}_d(\mathbf{V}) \\
		\mathbf{\Delta}_{\Gamma} \triangleq\!\!\!\!\!\! &(\mathbf{DV}_h)\!\times \!\mathbf{\Omega} + \mathbf{P}_h \!\times \!\mathbf{V}_h  - \mathbf{D}\dot{\mathbf{V}}_h  + \mathbf{\Gamma}_d\vspace{-0.1cm}
		\end{array} %\vspace{-0.1cm}
		\]
	\end{block}
\end{columns}
	
\end{frame}


\begin{frame}{Homography-based visual control}
\begin{columns}
	\column[content...]{0.4\textwidth}
	\begin{figure}
		\includegraphics[width = 50mm]{Images/Notation.png}
	\end{figure}\vspace{-0.5cm}
	$$\mathbf{\chi}^\star = \mathbf{p}_c + \mathbf{R}\mathbf{\chi} \Rightarrow \mathbf{\chi} = \mathbf{H}\mathbf{\chi}^\star$$ \vspace{-0.8cm}	
	\begin{block}{Homography}
		\vspace{-0.3cm}
		\begin{equation*}
		\mathbf{H} \triangleq \mathbf{R}^{\!\top} - \frac{1}{d^\star} \mathbf{R}^{\!\top} \mathbf{p}_C \mathbf{n}^{\star\!\top} 
		\end{equation*}
	\end{block}
	\column[content...]{0.5\textwidth}
%	\begin{block}{Pose decomposition}
%		\begin{itemize}
%			\item No need!
%		\end{itemize}
%	\end{block}

	\begin{block}{Visual errors}
		\[
		\begin{array}{rl}
			\mathbf{e}_p &\triangleq (\mathbf{I}_3 - \mathbf{H})\mathbf{m}^\star\\
			\mathbf{e}_\Theta &\triangleq \mathrm{vex}(\mathbf{H}^\top - \mathbf{H}) 
		\end{array}
		\]		
		with $\mathbf{m}^\star \in S^2$ satisfying: \vspace{-0.3cm}
		\[
		\begin{array}{ll} 
			|\mathbf{m}^\star| &= 1 \\
			\mathbf{n}^{\star \top} \mathbf{m}^\star &> 0
		\end{array}
		\]
		 	
	\end{block}
	\begin{block}{Control objective}
		\[
		\begin{array}{rl}
		(\mathbf{R}, \mathbf{p}_c) & \longrightarrow (\mathbf{I}_3, \mathbf{0})\\
		\mathbf{H} & \longrightarrow \mathbf{I}_3\\
		(\mathbf{e}_p, \mathbf{e}_\Theta) & \longrightarrow (\mathbf{0}, \mathbf{0})
		\end{array}
		\]	
	\end{block}
	
\end{columns}

\end{frame}

\begin{frame}{Kinematic control, recalled}
	\begin{block}{S.Benhimane and E.Malis, 2007}
		Homography-based 2D visual tracking and servoing
	\end{block}
	\begin{columns}
		\column[placement]{0.5\textwidth}
		\begin{block}{Lemma 1}
			The kinematic control law
			\vspace{-0.2cm}
			\begin{equation*}\label{maliscontrol}
			\mathbf{V}_{C} = -k_p \mathbf{e}_p\,,\quad \mathbf{\Omega} = -k_\Theta \mathbf{e}_\Theta 
			\vspace{-0.2cm} 
			\end{equation*}
			with $k_p, k_\Theta>0$, ensures the local exponential stability of $\mathbf{H}  \longrightarrow \mathbf{I}_3$.
		\end{block}
		\column[placement]{0.5\textwidth}
		\begin{block}{Advantage}
			\begin{itemize}
				\item No need homography decomposition				
			\end{itemize}
		\end{block}
		
		\begin{block}{Issues}
			\begin{itemize}
				\item Only kinematic control
				\item $\dot{\mathbf{e}}_p$ unknown
			\end{itemize}
		\end{block}
	\end{columns}
\end{frame}




\begin{frame}{Inner-outer loop control architecture}
%Block diagram of the proposed HBVS controller
\begin{figure}
	\includegraphics[width = 90mm]{Images/Block_diagram_2.png}
\end{figure}


\begin{itemize}
	\item Inner loop:
	\begin{itemize}
	
		\item $\mathbf{\Gamma}_C$ ensures  
		$(\mathbf{\Omega}, \mathbf{R}^\top \mathbf{e}_3) \longrightarrow (\mathbf{\Omega}_r, \mathbf{e}_3)$\\
		where: $ \mathbf{\Omega}_r \stackrel{\triangle}{=}  k_g \mathbf{e}_3 \times \mathbf{R}^{T} \mathbf{e}_3 + \omega_{3r} \mathbf{e}_3$ 
	\end{itemize} 
	\item Outer loop:
		\begin{itemize}
			\item $\mathbf{F}_c$ ensures $(\mathbf{e}_p, \mathbf{V}) \longrightarrow (\mathbf{0},\mathbf{0})$
			\item $\mathbf{F}_c$ together with $\mathbf{\Omega}_{r}$ ensure $\mathbf{H} \longrightarrow \mathbf{I}_3$		
		\end{itemize}
\end{itemize}
\end{frame}

%\begin{frame}{Outer-loop control design}
%%	\begin{columns}
%%		\column[content...]{0.5\textwidth}
%%		\begin{block}{Proposition 1}
%%			content...
%%		\end{block}
%%		\column[placement]{0.5\textwidth}
%%		\begin{block}{Proposition 3}
%%			content...
%%		\end{block}
%%	\end{columns}
%	
%	Consider the dynamics of $\mathbf{e}_p$:$$\dot {\mathbf{e}}_p = -\mathbf{\Omega}\times  \mathbf{e}_p   + a^\star \mathbf{V} + \boldsymbol{\varepsilon}(t), \boldsymbol{\varepsilon}(t) \text{ bounded and}  \longrightarrow 0$$
%	
%	 Consider the translational dynamics with $\mathbf{\Delta}_F \equiv \mathbf{0}$.
%	Introduce the augmented system \vspace{-0.1cm}
%	\begin{equation}\label{dot:ephatZD}
%	\dot {\hat{\mathbf{e}}}_p = -\mathbf{\Omega}\times  \hat{\mathbf{e}}_p   -\mathbf{K}_1 \hat{\mathbf{e}}_p + \mathbf{K}_1 {\mathbf{e}}_p, \quad  \hat{\mathbf{e}}_p(0) \in \mathbb{R}^3\vspace{-0.1cm}
%	\end{equation}
%	with $\mathbf{K}_1\in \mathbb{R}^{3\times 3}$ positive matrix. Apply the control force \vspace{-0.1cm}
%	\begin{equation}\label{u:force}
%	\mathbf{F}_c \!=\! \bar m \mathbf{M}^{-1} \big(\mathrm{sat}^{\eta_1}(k_2 \tilde{\mathbf{e}}_p)  - \mathrm{sat}^{\eta_2}( k_3 {\mathbf{e}}_p) \big) - \mathbf{F}_{gb} \vspace{-0.1cm}
%	\end{equation}
%	with positive numbers $\bar m$, $k_2$, $k_3$, $\eta_1$, $\eta_2$, and $\tilde {\mathbf{e}}_p \triangleq \hat{\mathbf{e}}_p - \mathbf{e}_p$, and $\mathrm{sat}^{(\cdot)}(\cdot)$ the classical saturation function.
%	Assume that $\mathbf{\Omega}$ remains bounded for all time. Then, the equilibrium $(\mathbf{e}_p,{\hat{\mathbf{e}}}_p, \mathbf{V})=(\mathbf{0},\mathbf{0},\mathbf{0})$ is globally asymptotically stable (GAS). Moreover, $\mathbf{F}_c$ remains bounded by \vspace{-0.1cm}
%	\begin{equation}\label{bound:force}
%	|\mathbf{F}_c(t)| \leq \bar{m} \underline{\lambda}_{\mathbf{M}}^{-1}(\eta_1 + \eta_2) + |mg - F_b| \vspace{-0.1cm}
%	\end{equation}
%	with $\underline{\lambda}_{\mathbf{M}}$ the smallest eigenvalue of $\mathbf{M}$.	
%
%\end{frame}

\begin{frame}{Outer-loop control design, $\mathbf{F}_c$}
	\begin{block}{Proposition 1, $\mathbf{v}_f = \mathbf{0}$}
		\begin{itemize}
			\item There exist: \\
			\begin{itemize}
				\item $\dot {\mathbf{e}}_p = -\mathbf{\Omega}\times  \mathbf{e}_p   + \frac{(\mathbf{n}^{\star\top}\mathbf{m}^\star)}{d^\star} \mathbf{V} + \boldsymbol{\varepsilon}(t), \boldsymbol{\varepsilon}(t) \text{ bounded and}  \longrightarrow 0$
				\item 
				$\mathbf{M}\dot{\mathbf{V}} = (\mathbf{M}\mathbf{V}) \!\times\!\mathbf{\Omega} + \mathbf{F}_c + \mathbf{F}_{gb} + \mathbf{F}_d(\mathbf{V}), \mathbf{\Delta}_F = 0 $ 
			\end{itemize}
			\item Introduce: \\
			\begin{itemize}
				\item $\dot {\hat{\mathbf{e}}}_p = -\mathbf{\Omega}\times  \hat{\mathbf{e}}_p   -\mathbf{K}_1 \hat{\mathbf{e}}_p + \mathbf{K}_1 {\mathbf{e}}_p, \quad  \hat{\mathbf{e}}_p(0) \in \mathbb{R}^3$
				\item
%				$\mathbf{F}_c \!=\! \bar m \mathbf{M}^{-1} \big(\mathrm{sat}^{\eta_1}(k_2 \tilde{\mathbf{e}}_p)  - \mathrm{sat}^{\eta_2}( k_3 {\mathbf{e}}_p) \big) - \mathbf{F}_{gb}$ 
				$\mathbf{F}_c \!=\! \bar m \mathbf{M}^{-1} \big(k_2(\hat{\mathbf{e}}_p - \mathbf{e}_p)  -  k_3 {\mathbf{e}}_p \big) - \mathbf{F}_{gb}$ 
			\end{itemize}
			\item Conclusion:\\ $(\mathbf{e}_p,{\hat{\mathbf{e}}}_p, \mathbf{V})=(\mathbf{0},\mathbf{0},\mathbf{0})$  globally asymptotically stable (GAS)			 
		\end{itemize}		
	\end{block}	
\end{frame}

\begin{frame}{Outer-loop control design, $\omega_{3r}$}
	\begin{block}{Proposition 3, $\mathbf{v}_f = \mathbf{0}$:}
		\begin{itemize}
			\item If:
			\begin{itemize}
				\item $\mathbf{\Gamma}_c$ ensures almost-GAS of the equilibrium $(\mathbf{\Omega}, \mathbf{R}^\top \mathbf{e}_3) = (\mathbf{\Omega}_r, \mathbf{e}_3)$, with $\mathbf{\Omega}_r$ defined by $$\mathbf{\Omega}_r \triangleq k_g \mathbf{e}_3 \times \mathbf{R}^\top \mathbf{e}_3 + \omega_{3r} \mathbf{e}_3, k_g >0  $$
				$$\dot{\omega}_{3r} = -k_{\Theta 2} \omega_{3r} - k_{\Theta 1} \mathrm{sat}^{\Delta_\Theta}(h_{1,2})$$
				 $$\quad \omega_{3r}(0) \in \mathbb{R}, k_{\Theta 1}, k_{\Theta 2}, \Delta_\Theta, h_{1,2} > 0$$
				\item $\mathbf{F}_c$ given by Proposition 1
			\end{itemize}
			\item Then: $\mathbf{H} = \mathbf{I}_3$ is almost-GAS.
		\end{itemize}
	\end{block}	
\end{frame}

\begin{frame}{Inner-loop control design}
	%Since the rotation dynamics is fully-actuated and both $\mathbf{\Omega}$ and $\mathbf{R}^\top \mathbf{e}_3$ are measured, it is not too difficult to design an effective inner-loop torque control that ensures  $(\mathbf{\Omega}, \mathbf{R}^\top \mathbf{e}_3) \longrightarrow (\mathbf{\Omega}_r, \mathbf{e}_3)$
	
	\begin{itemize}
		\item Fully-actuated rotation dynamics
		\item Both $\mathbf{\Omega}$ and $\mathbf{R}^\top \mathbf{e}_3$ are measured \\
		$\Rightarrow$ NOT to difficult to design $\mathbf{\Gamma}_c$ ensures $(\mathbf{\Omega}, \mathbf{R}^\top \mathbf{e}_3) \longrightarrow (\mathbf{\Omega}_r, \mathbf{e}_3)$ 
	\end{itemize}
\end{frame}

\begin{frame}{Other solved problems}
%Practical issues:\\
%- Imprecise model parameters\\
%- Current leads to "Munk moment" cannot be ignored\\
%requires that control laws must be robust  <- integral action\\
%Need to estimate disturbance $\mathbf{\Delta}_{\Gamma}$

	\begin{itemize}
		\item Practical issues: \\
		- Imprecise model parameters\\
		- "Munk moment" cannot be ignored due to current $\mathbf{v}_f \neq \mathbf{0}$
		\item Solutions:\\
		- Add integrator\\
		- Estimate disturbance $\mathbf{\Delta}_{\Gamma}$		
	\end{itemize}
 
\end{frame}


\section{Simulation results}
\begin{frame}{Simulation results, \tiny$\mathbf{p}_C(0) = [-2, -1.5, -1 ]^{\top} (m)$, $\mathbf{R}(0) = \mathbf{R}_{\{\frac{\pi}{18},-\frac{\pi}{18},\pi\}}$, $\mathbf{V}(0) = \mathbf{\Omega}(0) = \mathbf{0}$}
	\begin{columns}
		\column[content...]{0.55\textwidth}
		\begin{block}{Current $\mathbf{V}_f = \mathbf{0}$}
			\begin{figure}
				\includegraphics[width = 58mm]{Images/Data_sansCurrent_pos_RollPitch_Yaw2.pdf}\\
				\includegraphics[width = 58mm]{Images/Data_sansCurrent_Fc_Gc2.pdf}
			\end{figure}
		\end{block}	
		\column[content...]{0.55\textwidth}
		\begin{block}{Current $\mathbf{v}_f = [\frac{1}{2\sqrt{2}}, \frac{1}{2\sqrt{2}}, 0 ]^{\top} (m/s)$}
			\begin{figure}
				\includegraphics[width = 58mm]{Images/Data_avecCurrent_pos_RollPitch_Yaw2.pdf}\\
				\includegraphics[width = 58mm]{Images/Data_avecCurrent_Fc_Gc2.pdf}
			\end{figure}
		\end{block}	
	\end{columns}
\end{frame}



\section{Conclusion}

\begin{frame}{Conclusion}

  % Keep the summary *very short*.
  A new inertial-aided image-based visual servo controller for the stabilisation of compacted fully actuated AUVs:
  \begin{itemize}
  \item
    %The \alert{first main message} of your talk in one or two lines.
    Using image-based homography without its decomposition.
  \item
    Without relying on linear velocity measurement
  \item
    Ensures locally exponential stability, with robustness to unmodeled dynamics and disturbance.
  \end{itemize}
  
  % The following outlook is optional.
  \vskip0pt plus.5fill
  \begin{itemize}
  \item
    A testing campaign is on-going.    
  \end{itemize}
\end{frame}


\begin{frame}
\centering {
	{\huge \textbf{Thank you for your attention!}}	
	{\huge \textbf{Q\&A}}}

\end{frame}



% All of the following is optional and typically not needed. 
%\appendix
%\section<presentation>*{\appendixname}
%\subsection<presentation>*{For Further Reading}
%
%\begin{frame}[allowframebreaks]
%  \frametitle<presentation>{For Further Reading}
%    
%  \begin{thebibliography}{10}
%    
%  \beamertemplatebookbibitems
%  % Start with overview books.
%
%  \bibitem{Author1990}
%    A.~Author.
%    \newblock {\em Handbook of Everything}.
%    \newblock Some Press, 1990.
% 
%    
%  \beamertemplatearticlebibitems
%  % Followed by interesting articles. Keep the list short. 
%
%  \bibitem{Someone2000}
%    S.~Someone.
%    \newblock On this and that.
%    \newblock {\em Journal of This and That}, 2(1):50--100,
%    2000.
%  \end{thebibliography}
%\end{frame}

\end{document}


